\documentclass[a4paper]{article}

% \usepackage{T1}[fontenc]
% \usepackage{lmodern}
% \usepackage{fonstspec}
% \setmainfont{FreeSerif}

\renewcommand{\familydefault}{\rmdefault}
% \oddsidemargin = 2in

\usepackage{graphicx}
\usepackage{caption}
\usepackage{subcaption}
\usepackage[french, english]{babel}
\graphicspath{ {./img/} }
\selectlanguage{french}
\usepackage{algorithmicx}
% \usepackage{pseudocode}
\usepackage{algpseudocode}
\usepackage{algorithm}

%%%%%%%% FONT ATTEMPTS %%%%%%%%%%%%%%

% PLAY FAIR DISPLAY %%%%%%%%%%%%%%%%%%%%%
% \usepackage{PlayfairDisplay} % Use option 'black' to use the black
% %% variant for boldface
% %% Use option 'osf' to use oldstyle figures as default
% \renewcommand*\oldstylenums[1]{{\playfairOsF #1}}
% \usepackage[T1]{fontenc}
%%%%%%%%%%%%%%%%%%%%%%%%%%%%%%%%%%%%%%%%%%%%%%%%%%%%%%%%%


% %  tgschola %%%%%%%%%%%%%%%%%%%%%%%%%%%%
% \usepackage{tgschola}
% \usepackage[T1]{fontenc}
%%%%%%%%%%%%%%%%%%%%%%%%%%%%%%%%%%%%%%%%%%%%%%%%%%%%%%%%%


% NEW COMPUTER MODERN ROMAN %%%%%%%%%%%%%%%%%%%%%%%%%%%%%%%%%%%%%%%%55
% \setmathfont{NewCMMath-Regular.otf}
%%%%%%%%%%%%%%%%%%%%%%%%%%%%%%%%%%%%%%%%%%%%%%%%%%%%%%%%%

% DIDEROT %%%%%%%%%%%%%%%%%%%%%%%%%%%%%%%%5
% \usepackage{TheanoDidot}
% \usepackage[T1]{fontenc}
%%%%%%%%%%%%%%%%%%%%%%%%%%%%%%%%%%%%%%%%%%%%%%%%%%%%%%%%%

% GSDIDEROT
% \usepackage{gfsdidot}

% \fontfamily{lmr}\selectfont

% \usepackage{exercice}
\usepackage{amsmath}
\usepackage{bbm}
\usepackage{dsfont}
\usepackage{amsfonts}
\usepackage{amsthm}
\usepackage{mathtools}
\usepackage{gensymb}
\usepackage[margin=1in]{geometry}
\usepackage{tabto}
% \usepackage{titlesec}
\usepackage{fancyhdr}
\usepackage{listings}
\usepackage{xcolor}

\definecolor{codegreen}{rgb}{0,0.6,0}
\definecolor{codegray}{rgb}{0.5,0.5,0.5}
\definecolor{codepurple}{rgb}{0.58,0,0.82}
\definecolor{backcolour}{rgb}{0.95,0.95,0.92}

\lstdefinestyle{mystyle}{
    backgroundcolor=\color{backcolour},
    commentstyle=\color{codegreen},
    keywordstyle=\color{magenta},
    numberstyle=\tiny\color{codegray},
    stringstyle=\color{codepurple},
    basicstyle=\ttfamily\footnotesize,
    breakatwhitespace=false,
    breaklines=true,
    captionpos=b,
    keepspaces=true,
    numbers=left,
    numbersep=5pt,
    showspaces=false,
    showstringspaces=false,
    showtabs=false,
    tabsize=2
}

\lstset{style=mystyle}

\geometry{rmargin=1.5in}


% \titleformat*{\section}{\small\bfseries}
% \titleformat*{\subsection}{\Large\bfseries}
% \titleformat*{\subsubsection}{\large\bfseries}
% \titleformat*{\paragraph}{\large\bfseries}
% \titleformat*{\subparagraph}{\large\bfseries}

\makeatletter
\renewcommand\section{\@startsection{section}{1}{\z@}%
                                  {-3.5ex \@plus -1ex \@minus -.2ex}%
                                  {2.3ex \@plus.2ex}%
                                  {\normalfont\normalsize\bfseries}}
\makeatletter


% \makeatletter
% \@date
% \@title
% \makeatother

\DeclarePairedDelimiter\ceil{\lceil}{\rceil}
\DeclarePairedDelimiter\floor{\lfloor}{\rfloor}



\title{\Large \raggedright \bf Travaux Dirig\'es n\degree2}
% \author{\raggedright VOYLES Evan}
\date{}

% \newtheorem{exercice}{exercice}[section]
\newtheorem{exercice}{Exercice}
\newtheorem*{exercice*}{Exercice bonus}
\theoremstyle{definition}
\newtheorem{definition}{Definition}
\newtheorem{lemme}{Lemme}
% \renewcommand{\section*{}}

\fancyhf{}
\rhead{VOYLES Evan}
\lhead{MAIN}

\begin{document}
\pagestyle{fancy}

% \begin{titlepage}
    %     \maketitle
    %     \today
    % \end{titlepage}

    \makeatletter
    \maketitle
    \vspace{-.65in}
    \noindent
    MAIN3\\
    VOYLES Evan\\
    \today
    \makeatother

    \newenvironment{norm}{

        \normalfont


        }

        \newcommand{\Entree}{\textbf{Donn\'ees d'entr\'ee :  }}
        \newcommand{\Sortie}{\\\textbf{Donn\'ees de sortie :  }}

\begin{algorithm}[h!]
    \caption{Renvoi l'indice du plus petit \'element d'un tableau d'entiers}\label{alg:cap}
    \Entree T (tableau d'entiers), N (nombre entier, taille de T)
    \Sortie indice\_min (nombre entier)
    \begin{algorithmic}[1]
    \State $indice\_min \gets$ -1
    \State $min \gets T_1$
    \For{$i \gets 1, N$}
    \If{$T_i < min$}
    \State $min \gets T_i$
    \State $indice\_min \gets i$
    \EndIf
    \EndFor
    \State \Return $indice\_min$
    \end{algorithmic}
\end{algorithm}

\begin{algorithm}[h!]
    \caption{Nombre d'occurences}\label{alg:cap}
    \Entree T (tableau d'entiers), n (nombre entier), P (nombre entier, taille de T)
    \Sortie nombre\_fois (nombre entier)
    \begin{algorithmic}[1]
    \State $nombre\_fois \gets 0$
    \For{$i \gets 1, P$}
    \If{$T_i = n $}
    \State $nombre\_fois \gets nombre\_fois + 1$
    \EndIf
    \EndFor
    \State \Return $nombre\_fois$
    \end{algorithmic}
\end{algorithm}

\begin{algorithm}[h!]
    \caption{Moyenne et ecart-type}\label{alg:cap}
    \Entree T (tableau de doubles), N (nombre entier, taille de T)
    \Sortie moyenne (double), ecart\_type (double)
    \begin{algorithmic}[1]
    \State $moyenne \gets 0$
    \State $ ecart\_type \gets 0$
    \State $sum \gets 0$
    \State $ecart\_carre \gets 0$
    \For{$i \gets 1, N$}
    \State $sum \gets sum + T_i$
    \EndFor
    \State $moyenne \gets sum/N$
    \For{$i \gets 1, N$}
    \State $ecart\_carre \gets (T_i - moyenne)^2$
    \EndFor
    \State $ecart\_type \gets \sqrt{ecart\_carre/(N-1)}$
    \State \Return $moyenne, ecart\_type$
    \end{algorithmic}
\end{algorithm}
\newpage
\begin{algorithm}[t!]
    \caption{Palindrome}\label{alg:cap}
    \Entree T (tableau d'entiers), N (nombre entier, taille de T)
    \Sortie est\_palindrome (boolean)
    \begin{algorithmic}[1]
    \For{$i \gets 1, \floor*{N/2}$}
    \If{$T_i \neq T_{N-i+1}$}
    \State \Return $false$
    \EndIf
    \EndFor
    \State \Return $true$
    \end{algorithmic}
\end{algorithm}

\begin{algorithm}[t!]
    \caption{Remplacement des doublons}\label{alg:cap}
    \Entree T (tableau d'entiers $> 0$), N (nombre entier, taille de T)
    \Sortie T avec les doublons remplace par 0
    \begin{algorithmic}[1]
    \State $old\_value \gets T_1$
    \For{$i \gets 2,N$}
    \If{$T_i = old\_value$}
    \State $T_i \gets 0$
    \Else
    \State $old\_value \gets T_i$
    \EndIf
    \EndFor
    \State \Return $T$
    \end{algorithmic}
\end{algorithm}

\setcounter{algorithm}{6}
\begin{algorithm}[h!]
    \caption{Crible d'\'Erastosth\`ene}\label{alg:cap}
    \Entree T (tableau d'entiers $> 0$), N (nombre entier)
    \Sortie T avec les \'el\'ements non-primes remplac\'es par $-1$
    \begin{algorithmic}[1]
    \State $T \gets 2:N$
    \For{$i \gets 2,\floor*{\sqrt{N}}$}
    \For{$j \gets 1, N$}
    \If{$T_j \bmod i \equiv 0$}
    \State $T_j \gets -1$
    \EndIf
    \EndFor
    \EndFor
    \State \Return $T$
    \end{algorithmic}
\end{algorithm}
\newpage

\newenvironment{blank}{}

\begin{blank}
    \Large \raggedright \bf Travaux Dirig\'es n\degree3
\end{blank}
\setcounter{algorithm}{0}

\begin{algorithm}[h!]
    \caption{Crible d'\'Erastosth\`ene}\label{alg:cap}
    \Entree T (tableau d'entiers $> 0$), N (nombre entier)
    \Sortie T avec les \'el\'ements non-primes remplac\'es par $-1$
    \begin{algorithmic}[1]
    \State $T \gets 2:N$
    \For{$i \gets 2,\floor*{\sqrt{N}}$}
    \For{$j \gets 1, N$}
    \If{$T_j \bmod i \equiv 0$}
    \State $T_j \gets -1$
    \EndIf
    \EndFor
    \EndFor
    \State \Return $T$
    \end{algorithmic}
\end{algorithm}

\begin{blank}
    \Large \raggedright \bf Travaux Dirig\'es n\degree4
\end{blank}
\setcounter{algorithm}{0}

\setcounter{exercice}{2}
\begin{exercice}
    On consid\`ere le programe suivant:
\end{exercice}

\begin{lstlisting}[language=C]
#include <stdio.h>
int a = 27;
int chose(int a);
int machin();

int chose(int a)
{
    return a+17+machin(); // a est une variable locale
}
int machin()
{
    return a; // a est une variable globale
}
int main()
{
    int a=1; // nouvelle declaration de a cache le 'a' de ligne 2
    a=chose(a); // a <- (1 + 17 + 27)
    printf("%d\n", a);
}
\end{lstlisting}
Affiche 45 a l'\'ecran.

\begin{exercice}
    D\'efinissez l'affichage produit par l'ex\'ecution du programme. Expliquez.
\end{exercice}
\begin{lstlisting}[language=C]
#include <stdio.h>

int nb = 3;
//============================================================

void mal_ecrit_1 (int nb) // nb est une variable locale
{
    printf ("nb mal_ecrit_1 = %d\n",nb++); // nb++ renvoie nb et s'incremente APRES
    {
        int nb = 14; // cette exo est deguelasse
        printf ("nb mal_ecrit_1 = %d\n",nb); // nb = 14
    }
    printf ("nb mal_ecrit_1 = %d\n",nb); // nb locale
}
//============================================================

int mal_ecrit_2 (int x, int y) // nb dans cette scope est globale
{
    printf ("nb mal_ecrit_2 = %d\n",x + nb);
    printf ("nb mal_ecrit_2 = %d\n",y + nb);
    return (nb *= 0); // nb <- 0
}
//============================================================

int main(void)
{
    int x = 3;
    int y = 6;
    printf ("nb main = %d\n",nb); // nb globale est initialise a 3
    mal_ecrit_1(nb); // ne modifie pas du tout le nb globale
    printf ("nb main = %d\n",nb);
    printf("resultat de mal_ecrit_2 = %d\n",mal_ecrit_2(y,x)); // utilise nb = 3, renvoie nb <- 0
    // Attention, on a appele la fonction mal_ecrit_2 avec l'ordre de y et x inverse
    printf ("nb = %d\n",nb);
    return 0;
}
\end{lstlisting}

\noindent \textbf{29:} \verb|nb main = 3|\\
\textbf{30:} \verb|nb mal_ecrit_1 = 3|\\
\verb|    nb mal_ecrit_1 = 14| \\
\verb|    nb mal_ecrit_1 = 4| \\
\textbf{31:} \verb|nb main = 3| \\
\textbf{32:} \verb|nb mal_ecrit_2 = 9| \\
\verb|    nb mal_ecrit_2 = 6|\\
\verb|    resultat de mal_ecrit_2 = 0|\\
\textbf{34:} \verb|nb = 0|

\begin{blank}
    \Large \raggedright \bf Travaux Dirig\'es n\degree5
\end{blank}

\begin{blank}
    \Large \raggedright \bf Travaux Dirig\'es n\degree6
\end{blank}

\setcounter{exercice}{0}

\begin{exercice}
    Calculer $\pi$ en appliquant la suite de Leibniz : $\pi/4 = 1 - 1/3 + 1/5 - 1/7 + 1/9 - \dots$.
\end{exercice}

Une boucle \verb|while| me semble la plus appropri\'ee parce que on ne sait pas exactement le nombre de fois que l'on
devrait it\'erer l'alorithme.

\begin{algorithm}[h!]
    \caption{Crible d'\'Erastosth\`ene}\label{alg:cap}
    \Entree T (tableau d'entiers $> 0$), N (nombre entier)
    \Sortie T avec les \'el\'ements non-primes remplac\'es par $-1$
    \begin{algorithmic}[1]
    \State $T \gets 2:N$
    \For{$i \gets 2,\floor*{\sqrt{N}}$}
    \For{$j \gets 1, N$}
    \If{$T_j \bmod i \equiv 0$}
    \State $T_j \gets -1$
    \EndIf
    \EndFor
    \EndFor
    \State \Return $T$
    \end{algorithmic}
\end{algorithm}

\end{document}